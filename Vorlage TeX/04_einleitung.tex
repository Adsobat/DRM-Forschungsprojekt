\chapter{Einleitung}
\label{sec:einleitung}
\begin{spacing}{1.5}
Der Wandel vom Industriezeitalter zum Digitalzeitalter schreitet immer schneller voran. Dies macht sich nicht nur im geschäftlichen sondern auch im privaten Umfeld bemerkbar. Digitale Medien gewinnen mehr und mehr die Oberhand. Dies ruft gleichzeitig kriminelle Handlungen auf den Plan. 
Diese Arbeit versucht einen Lösungsweg für diese Problematik zu arbeiten und eine sichere Methode vorzustellen, welche Langfristig eine Sicherheit bietet.
So lässt sich durch Studium dieser Arbeit  

\end{spacing}


\section{Motivation}
\label{sec:Motivation}
\begin{spacing}{1.5}
Derzeit befinden sich auf dem Markt sehr viele proprietäre Systeme, welche DRM (Digital Rights Management) zum Schutz digitaler Medien anbieten. Allerdings sind solche Anwendungen zum einen sehr einschränkend für den Benutzer. Das beutet entweder die Beschränkung auf eine bestimmte Hardware oder die Einschränkung des Nutzungsverhaltens.  
\end{spacing}


\section{Ziel der Arbeit}
Ziel dieser Arbeit ist es ein Konzept aufzuzeigen welches einen Mehrwert für die Verlage und für den jeweiligen Nutzer bietet. Dabei werden die Stichworte Sicherheit und gleichzeitige Anonymität auf Basis des juristischen Kontextes beachtet. 
\label{sec:Ziel der Arbeit}
\begin{spacing}{1.5}

\end{spacing}


\section{Überblick}
\label{sec:Überblick}
\begin{spacing}{1.5}

\end{spacing}